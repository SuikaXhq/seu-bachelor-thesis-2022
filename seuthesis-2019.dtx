% \iffalse meta-comment
%
% Copyright (C) 2022 Haoqing Xu
% School of Artificial Intelligence, Southeast University.
% --------------------------------------------------------------------------
%
% This work may be distributed and/or modified under the
% conditions of the LaTeX Project Public License, either
% version 1.3c of this license or (at your option) any later
% version. This version of this license is in
%    http://www.latex-project.org/lppl/lppl-1-3c.txt
% and the latest version of this license is in
%    http://www.latex-project.org/lppl.txt
% and version 1.3 or later is part of all distributions of
% LaTeX version 2005/12/01 or later.
%
% This work has the LPPL maintenance status `maintained'.
% The Current Maintainer of this work is Haoqing Xu.
%
% Home Page of the Project: https://github.com/SuikaXhq/seu-bachelor-thesis-2019
%
% \fi
% \iffalse
%<*driver>
\ProvidesFile{seuthesis-2019.dtx}
%</driver>
%<cls>\NeedsTeXFormat{LaTeX2e}[2005/12/01]
%<cls>\ProvidesClass{seuthesis-2019}
%<cfg>\ProvidesFile{seuthesis-2019.cfg}
%<*cls>
  [2022.4.16 v0.1 LaTeX Template for SEU Bachelor Thesis ver2019]
%</cls>
%<*driver>
\documentclass[10pt,a4paper,oneside]{ltxdoc}
\usepackage{seuthesisdoc}
\EnableCrossrefs
\CodelineIndex
\RecordChanges
\begin{document}
  \DocInput{seuthesis-2019.dtx}
\end{document}
%</driver>
% \fi

% \CheckSum{0}
%
% \CharacterTable
%  {Upper-case    \A\B\C\D\E\F\G\H\I\J\K\L\M\N\O\P\Q\R\S\T\U\V\W\X\Y\Z
%   Lower-case    \a\b\c\d\e\f\g\h\i\j\k\l\m\n\o\p\q\r\s\t\u\v\w\x\y\z
%   Digits        \0\1\2\3\4\5\6\7\8\9
%   Exclamation   \!     Double quote  \"     Hash (number) \#
%   Dollar        \$     Percent       \%     Ampersand     \&
%   Acute accent  \'     Left paren    \(     Right paren   \)
%   Asterisk      \*     Plus          \+     Comma         \,
%   Minus         \-     Point         \.     Solidus       \/
%   Colon         \:     Semicolon     \;     Less than     \<
%   Equals        \=     Greater than  \>     Question mark \?
%   Commercial at \@     Left bracket  \[     Backslash     \\
%   Right bracket \]     Circumflex    \^     Underscore    \_
%   Grave accent  \`     Left brace    \{     Vertical bar  \|
%   Right brace   \}     Tilde         \~}
% 
% \StopEventually{\PrintIndex}

% \section{实现细节}
% \subsection{字段定义}
% 定义模版内常用字段,包括章节标题、固定项名称等。该部分保存于文档配置文件(*.cfg)。变量命名规则为seut@功能@类型@名称,功能包含预设字段(field)、变量(var)等,类型包括字符串(str)、数值(num)、布尔(bool)等,名称可能包含所属页面等辅助信息。
% 
% 定义常用变量。
%    \begin{macrocode}
%<*cfg>
\newcommand*{\seut@space}[1]{\hspace{#1\ccwd}}
%    \end{macrocode}
% 定义封面页字段。
%    \begin{macrocode}
\newcommand*{\seut@const@str@titlepage@logotitle}     {毕业设计(论文)报告}
\newcommand*{\seut@field@str@titlepage@title}         {题\seut@space{0.5}目:}
\newcommand*{\seut@field@str@titlepage@studentnum}    {学\seut@space{2}号:}
\newcommand*{\seut@field@str@titlepage@author}        {姓\seut@space{2}名:}
\newcommand*{\seut@field@str@titlepage@department}    {学\seut@space{2}院:}
\newcommand*{\seut@field@str@titlepage@major}         {专\seut@space{2}业:}
\newcommand*{\seut@field@str@titlepage@supervisor}    {指导教师:}
\newcommand*{\seut@field@str@titlepage@period}        {起止日期:}

\newif\ifseut@title@long\seut@title@longtrue
\newcommand*{\seut@var@str@titlepage@title}{}
\newcommand*{\seut@var@str@titlepage@title@up}{}
\newcommand*{\seut@var@str@titlepage@title@down}{}
\newcommand*{\seut@var@str@titlepage@studentnum}{}
\newcommand*{\seut@var@str@titlepage@author}{}
\newcommand*{\seut@var@str@titlepage@department}{}
\newcommand*{\seut@var@str@titlepage@major}{}
\newcommand*{\seut@var@str@titlepage@supervisor}{}
\newcommand*{\seut@var@str@titlepage@period}{}
%    \end{macrocode}
% 定义声明页字段。
%    \begin{macrocode}
\newcommand*{\seut@const@str@declare@origin@title}{东南大学毕业(设计)论文独创性声明}
\newcommand*{\seut@const@str@declare@origin@content}{%
  本人声明所呈交的毕业(设计)论文是我个人在导师指导下进行的研究工作及取得的研究成果。尽我所知,%
  除了文中特别加以标注和致谢的地方外,论文中不包含其他人已经发表或撰写过的研究成果,也不包含为获%
  得东南大学或其它教育机构的学位或证书而使用过的材料。与我一同工作的同志对本研究所做的任何贡献均%
  已在论文中作了明确的说明并表示了谢意。%
}
\newcommand*{\seut@field@str@declare@authorsign}{论文作者签名:}
\newcommand*{\seut@field@str@declare@date}{日期:}
\newcommand*{\seut@field@str@declare@supervisorsign}{导师签名:}
\newcommand*{\seut@const@str@declare@usage@title}{东南大学毕业(设计)论文使用授权声明}
\newcommand*{\seut@const@str@declare@usage@content}{%
  东南大学有权保留本人所送交毕业(设计)论文的复印件和电子文档,可以采用影印、缩印或其他复制手段%
  保存论文。本人电子文档的内容和纸质论文的内容相一致。除在保密期内的保密论文外,允许论文被查阅和%
  借阅,可以公布(包括刊登)论文的全部或部分内容。论文的公布(包括刊登)授权东南大学教务处办理。%
}
%    \end{macrocode}
% 定义字符串常量。
%    \begin{macrocode}
\newcommand*{\seut@const@str@headertext}{东南大学本科毕业设计(论文)}
%    \end{macrocode}
% 定义英文字体名称,包括衬线、无衬线、等宽字体。
%    \begin{macrocode}
\newcommand*{\seut@const@str@font@serif@win}        {Times New Roman}
\newcommand*{\seut@const@str@font@sans@win}         {Arial}
\newcommand*{\seut@const@str@font@mono@win}         {Courier New}
%    \end{macrocode}
% 定义中文字体名称,包括宋体、黑体、仿宋、楷体四种常用字体。
%    \begin{macrocode}
\newcommand*{\seut@const@str@font@songti@FZ}        {FZShuSong-Z01}
\newcommand*{\seut@const@str@font@heiti@FZ}         {FZHei-B01}
\newcommand*{\seut@const@str@font@fangsong@FZ}      {FZFangSong-Z02}
\newcommand*{\seut@const@str@font@kaiti@FZ}         {FZKai-Z03}

\newcommand*{\seut@const@str@font@songti@win}       {SimSun}
\newcommand*{\seut@const@str@font@heiti@win}        {SimHei}
\newcommand*{\seut@const@str@font@fangsong@win}     {FangSong}
\newcommand*{\seut@const@str@font@kaiti@win}        {KaiTi}
%    \end{macrocode}
% 定义字体选项布尔值,默认为Windows字体。
%    \begin{macrocode}
\newif\ifseut@font@win\seut@font@wintrue
\newif\ifseut@font@mac\seut@font@macfalse
%    \end{macrocode}
%</cfg>

% \subsection{启动前检查}
% 检查\XeLaTeX引擎。
%    \begin{macrocode}
%<*cls>
\RequirePackage{ifxetex}
\RequireXeTeX
%    \end{macrocode}

% \subsection{文档类选项处理}
% 导入配置文件。
%    \begin{macrocode}
% \iffalse meta-comment
%
% Copyright (C) 2022 Haoqing Xu
% School of Artificial Intelligence, Southeast University.
% --------------------------------------------------------------------------
%
% This work may be distributed and/or modified under the
% conditions of the LaTeX Project Public License, either
% version 1.3c of this license or (at your option) any later
% version. This version of this license is in
%    http://www.latex-project.org/lppl/lppl-1-3c.txt
% and the latest version of this license is in
%    http://www.latex-project.org/lppl.txt
% and version 1.3 or later is part of all distributions of
% LaTeX version 2005/12/01 or later.
%
% This work has the LPPL maintenance status `maintained'.
% The Current Maintainer of this work is Haoqing Xu.
%
% Home Page of the Project: https://github.com/SuikaXhq/seu-bachelor-thesis-2019
%
% \fi
% \iffalse
%<*driver>
\ProvidesFile{seuthesis-2019.dtx}
%</driver>
%<cls>\NeedsTeXFormat{LaTeX2e}[2005/12/01]
%<cls>\ProvidesClass{seuthesis-2019}
%<cfg>\ProvidesFile{seuthesis-2019.cfg}
%<*cls>
  [2022.4.16 v0.1 LaTeX Template for SEU Bachelor Thesis ver2019]
%</cls>
%<*driver>
\documentclass[10pt,a4paper,oneside]{ltxdoc}
\usepackage{seuthesisdoc}
\EnableCrossrefs
\CodelineIndex
\RecordChanges
\begin{document}
  \DocInput{seuthesis-2019.dtx}
\end{document}
%</driver>
% \fi

% \CheckSum{0}
%
% \CharacterTable
%  {Upper-case    \A\B\C\D\E\F\G\H\I\J\K\L\M\N\O\P\Q\R\S\T\U\V\W\X\Y\Z
%   Lower-case    \a\b\c\d\e\f\g\h\i\j\k\l\m\n\o\p\q\r\s\t\u\v\w\x\y\z
%   Digits        \0\1\2\3\4\5\6\7\8\9
%   Exclamation   \!     Double quote  \"     Hash (number) \#
%   Dollar        \$     Percent       \%     Ampersand     \&
%   Acute accent  \'     Left paren    \(     Right paren   \)
%   Asterisk      \*     Plus          \+     Comma         \,
%   Minus         \-     Point         \.     Solidus       \/
%   Colon         \:     Semicolon     \;     Less than     \<
%   Equals        \=     Greater than  \>     Question mark \?
%   Commercial at \@     Left bracket  \[     Backslash     \\
%   Right bracket \]     Circumflex    \^     Underscore    \_
%   Grave accent  \`     Left brace    \{     Vertical bar  \|
%   Right brace   \}     Tilde         \~}
% 
% \StopEventually{\PrintIndex}

% \section{实现细节}
% \subsection{字段定义}
% 定义模版内常用字段,包括章节标题、固定项名称等。该部分保存于文档配置文件(*.cfg)。变量命名规则为seut@功能@类型@名称,功能包含预设字段(field)、变量(var)等,类型包括字符串(str)、数值(num)、布尔(bool)等,名称可能包含所属页面等辅助信息。
% 
% 定义常用变量。
%    \begin{macrocode}
%<*cfg>
\newcommand*{\seut@space}[1]{\hspace{#1\ccwd}}
%    \end{macrocode}
% 定义封面页字段。
%    \begin{macrocode}
\newcommand*{\seut@const@str@titlepage@logotitle}     {毕业设计(论文)报告}
\newcommand*{\seut@field@str@titlepage@title}         {题\seut@space{0.5}目:}
\newcommand*{\seut@field@str@titlepage@studentnum}    {学\seut@space{2}号:}
\newcommand*{\seut@field@str@titlepage@author}        {姓\seut@space{2}名:}
\newcommand*{\seut@field@str@titlepage@department}    {学\seut@space{2}院:}
\newcommand*{\seut@field@str@titlepage@major}         {专\seut@space{2}业:}
\newcommand*{\seut@field@str@titlepage@supervisor}    {指导教师:}
\newcommand*{\seut@field@str@titlepage@period}        {起止日期:}

\newif\ifseut@title@long\seut@title@longtrue
\newcommand*{\seut@var@str@titlepage@title}{}
\newcommand*{\seut@var@str@titlepage@title@up}{}
\newcommand*{\seut@var@str@titlepage@title@down}{}
\newcommand*{\seut@var@str@titlepage@studentnum}{}
\newcommand*{\seut@var@str@titlepage@author}{}
\newcommand*{\seut@var@str@titlepage@department}{}
\newcommand*{\seut@var@str@titlepage@major}{}
\newcommand*{\seut@var@str@titlepage@supervisor}{}
\newcommand*{\seut@var@str@titlepage@period}{}
%    \end{macrocode}
% 定义声明页字段。
%    \begin{macrocode}
\newcommand*{\seut@const@str@declare@origin@title}{东南大学毕业(设计)论文独创性声明}
\newcommand*{\seut@const@str@declare@origin@content}{%
  本人声明所呈交的毕业(设计)论文是我个人在导师指导下进行的研究工作及取得的研究成果。尽我所知,%
  除了文中特别加以标注和致谢的地方外,论文中不包含其他人已经发表或撰写过的研究成果,也不包含为获%
  得东南大学或其它教育机构的学位或证书而使用过的材料。与我一同工作的同志对本研究所做的任何贡献均%
  已在论文中作了明确的说明并表示了谢意。%
}
\newcommand*{\seut@field@str@declare@authorsign}{论文作者签名:}
\newcommand*{\seut@field@str@declare@date}{日期:}
\newcommand*{\seut@field@str@declare@supervisorsign}{导师签名:}
\newcommand*{\seut@const@str@declare@usage@title}{东南大学毕业(设计)论文使用授权声明}
\newcommand*{\seut@const@str@declare@usage@content}{%
  东南大学有权保留本人所送交毕业(设计)论文的复印件和电子文档,可以采用影印、缩印或其他复制手段%
  保存论文。本人电子文档的内容和纸质论文的内容相一致。除在保密期内的保密论文外,允许论文被查阅和%
  借阅,可以公布(包括刊登)论文的全部或部分内容。论文的公布(包括刊登)授权东南大学教务处办理。%
}
%    \end{macrocode}
% 定义字符串常量。
%    \begin{macrocode}
\newcommand*{\seut@const@str@headertext}{东南大学本科毕业设计(论文)}
%    \end{macrocode}
% 定义英文字体名称,包括衬线、无衬线、等宽字体。
%    \begin{macrocode}
\newcommand*{\seut@const@str@font@serif@win}        {Times New Roman}
\newcommand*{\seut@const@str@font@sans@win}         {Arial}
\newcommand*{\seut@const@str@font@mono@win}         {Courier New}
%    \end{macrocode}
% 定义中文字体名称,包括宋体、黑体、仿宋、楷体四种常用字体。
%    \begin{macrocode}
\newcommand*{\seut@const@str@font@songti@FZ}        {FZShuSong-Z01}
\newcommand*{\seut@const@str@font@heiti@FZ}         {FZHei-B01}
\newcommand*{\seut@const@str@font@fangsong@FZ}      {FZFangSong-Z02}
\newcommand*{\seut@const@str@font@kaiti@FZ}         {FZKai-Z03}

\newcommand*{\seut@const@str@font@songti@win}       {SimSun}
\newcommand*{\seut@const@str@font@heiti@win}        {SimHei}
\newcommand*{\seut@const@str@font@fangsong@win}     {FangSong}
\newcommand*{\seut@const@str@font@kaiti@win}        {KaiTi}
%    \end{macrocode}
% 定义字体选项布尔值,默认为Windows字体。
%    \begin{macrocode}
\newif\ifseut@font@win\seut@font@wintrue
\newif\ifseut@font@mac\seut@font@macfalse
%    \end{macrocode}
%</cfg>

% \subsection{启动前检查}
% 检查\XeLaTeX引擎。
%    \begin{macrocode}
%<*cls>
\RequirePackage{ifxetex}
\RequireXeTeX
%    \end{macrocode}

% \subsection{文档类选项处理}
% 导入配置文件。
%    \begin{macrocode}
% \iffalse meta-comment
%
% Copyright (C) 2022 Haoqing Xu
% School of Artificial Intelligence, Southeast University.
% --------------------------------------------------------------------------
%
% This work may be distributed and/or modified under the
% conditions of the LaTeX Project Public License, either
% version 1.3c of this license or (at your option) any later
% version. This version of this license is in
%    http://www.latex-project.org/lppl/lppl-1-3c.txt
% and the latest version of this license is in
%    http://www.latex-project.org/lppl.txt
% and version 1.3 or later is part of all distributions of
% LaTeX version 2005/12/01 or later.
%
% This work has the LPPL maintenance status `maintained'.
% The Current Maintainer of this work is Haoqing Xu.
%
% Home Page of the Project: https://github.com/SuikaXhq/seu-bachelor-thesis-2019
%
% \fi
% \iffalse
%<*driver>
\ProvidesFile{seuthesis-2019.dtx}
%</driver>
%<cls>\NeedsTeXFormat{LaTeX2e}[2005/12/01]
%<cls>\ProvidesClass{seuthesis-2019}
%<cfg>\ProvidesFile{seuthesis-2019.cfg}
%<*cls>
  [2022.4.16 v0.1 LaTeX Template for SEU Bachelor Thesis ver2019]
%</cls>
%<*driver>
\documentclass[10pt,a4paper,oneside]{ltxdoc}
\usepackage{seuthesisdoc}
\EnableCrossrefs
\CodelineIndex
\RecordChanges
\begin{document}
  \DocInput{seuthesis-2019.dtx}
\end{document}
%</driver>
% \fi

% \CheckSum{0}
%
% \CharacterTable
%  {Upper-case    \A\B\C\D\E\F\G\H\I\J\K\L\M\N\O\P\Q\R\S\T\U\V\W\X\Y\Z
%   Lower-case    \a\b\c\d\e\f\g\h\i\j\k\l\m\n\o\p\q\r\s\t\u\v\w\x\y\z
%   Digits        \0\1\2\3\4\5\6\7\8\9
%   Exclamation   \!     Double quote  \"     Hash (number) \#
%   Dollar        \$     Percent       \%     Ampersand     \&
%   Acute accent  \'     Left paren    \(     Right paren   \)
%   Asterisk      \*     Plus          \+     Comma         \,
%   Minus         \-     Point         \.     Solidus       \/
%   Colon         \:     Semicolon     \;     Less than     \<
%   Equals        \=     Greater than  \>     Question mark \?
%   Commercial at \@     Left bracket  \[     Backslash     \\
%   Right bracket \]     Circumflex    \^     Underscore    \_
%   Grave accent  \`     Left brace    \{     Vertical bar  \|
%   Right brace   \}     Tilde         \~}
% 
% \StopEventually{\PrintIndex}

% \section{实现细节}
% \subsection{字段定义}
% 定义模版内常用字段,包括章节标题、固定项名称等。该部分保存于文档配置文件(*.cfg)。变量命名规则为seut@功能@类型@名称,功能包含预设字段(field)、变量(var)等,类型包括字符串(str)、数值(num)、布尔(bool)等,名称可能包含所属页面等辅助信息。
% 
% 定义常用变量。
%    \begin{macrocode}
%<*cfg>
\newcommand*{\seut@space}[1]{\hspace{#1\ccwd}}
%    \end{macrocode}
% 定义封面页字段。
%    \begin{macrocode}
\newcommand*{\seut@const@str@titlepage@logotitle}     {毕业设计(论文)报告}
\newcommand*{\seut@field@str@titlepage@title}         {题\seut@space{0.5}目:}
\newcommand*{\seut@field@str@titlepage@studentnum}    {学\seut@space{2}号:}
\newcommand*{\seut@field@str@titlepage@author}        {姓\seut@space{2}名:}
\newcommand*{\seut@field@str@titlepage@department}    {学\seut@space{2}院:}
\newcommand*{\seut@field@str@titlepage@major}         {专\seut@space{2}业:}
\newcommand*{\seut@field@str@titlepage@supervisor}    {指导教师:}
\newcommand*{\seut@field@str@titlepage@period}        {起止日期:}

\newif\ifseut@title@long\seut@title@longtrue
\newcommand*{\seut@var@str@titlepage@title}{}
\newcommand*{\seut@var@str@titlepage@title@up}{}
\newcommand*{\seut@var@str@titlepage@title@down}{}
\newcommand*{\seut@var@str@titlepage@studentnum}{}
\newcommand*{\seut@var@str@titlepage@author}{}
\newcommand*{\seut@var@str@titlepage@department}{}
\newcommand*{\seut@var@str@titlepage@major}{}
\newcommand*{\seut@var@str@titlepage@supervisor}{}
\newcommand*{\seut@var@str@titlepage@period}{}
%    \end{macrocode}
% 定义声明页字段。
%    \begin{macrocode}
\newcommand*{\seut@const@str@declare@origin@title}{东南大学毕业(设计)论文独创性声明}
\newcommand*{\seut@const@str@declare@origin@content}{%
  本人声明所呈交的毕业(设计)论文是我个人在导师指导下进行的研究工作及取得的研究成果。尽我所知,%
  除了文中特别加以标注和致谢的地方外,论文中不包含其他人已经发表或撰写过的研究成果,也不包含为获%
  得东南大学或其它教育机构的学位或证书而使用过的材料。与我一同工作的同志对本研究所做的任何贡献均%
  已在论文中作了明确的说明并表示了谢意。%
}
\newcommand*{\seut@field@str@declare@authorsign}{论文作者签名:}
\newcommand*{\seut@field@str@declare@date}{日期:}
\newcommand*{\seut@field@str@declare@supervisorsign}{导师签名:}
\newcommand*{\seut@const@str@declare@usage@title}{东南大学毕业(设计)论文使用授权声明}
\newcommand*{\seut@const@str@declare@usage@content}{%
  东南大学有权保留本人所送交毕业(设计)论文的复印件和电子文档,可以采用影印、缩印或其他复制手段%
  保存论文。本人电子文档的内容和纸质论文的内容相一致。除在保密期内的保密论文外,允许论文被查阅和%
  借阅,可以公布(包括刊登)论文的全部或部分内容。论文的公布(包括刊登)授权东南大学教务处办理。%
}
%    \end{macrocode}
% 定义字符串常量。
%    \begin{macrocode}
\newcommand*{\seut@const@str@headertext}{东南大学本科毕业设计(论文)}
%    \end{macrocode}
% 定义英文字体名称,包括衬线、无衬线、等宽字体。
%    \begin{macrocode}
\newcommand*{\seut@const@str@font@serif@win}        {Times New Roman}
\newcommand*{\seut@const@str@font@sans@win}         {Arial}
\newcommand*{\seut@const@str@font@mono@win}         {Courier New}
%    \end{macrocode}
% 定义中文字体名称,包括宋体、黑体、仿宋、楷体四种常用字体。
%    \begin{macrocode}
\newcommand*{\seut@const@str@font@songti@FZ}        {FZShuSong-Z01}
\newcommand*{\seut@const@str@font@heiti@FZ}         {FZHei-B01}
\newcommand*{\seut@const@str@font@fangsong@FZ}      {FZFangSong-Z02}
\newcommand*{\seut@const@str@font@kaiti@FZ}         {FZKai-Z03}

\newcommand*{\seut@const@str@font@songti@win}       {SimSun}
\newcommand*{\seut@const@str@font@heiti@win}        {SimHei}
\newcommand*{\seut@const@str@font@fangsong@win}     {FangSong}
\newcommand*{\seut@const@str@font@kaiti@win}        {KaiTi}
%    \end{macrocode}
% 定义字体选项布尔值,默认为Windows字体。
%    \begin{macrocode}
\newif\ifseut@font@win\seut@font@wintrue
\newif\ifseut@font@mac\seut@font@macfalse
%    \end{macrocode}
%</cfg>

% \subsection{启动前检查}
% 检查\XeLaTeX引擎。
%    \begin{macrocode}
%<*cls>
\RequirePackage{ifxetex}
\RequireXeTeX
%    \end{macrocode}

% \subsection{文档类选项处理}
% 导入配置文件。
%    \begin{macrocode}
% \iffalse meta-comment
%
% Copyright (C) 2022 Haoqing Xu
% School of Artificial Intelligence, Southeast University.
% --------------------------------------------------------------------------
%
% This work may be distributed and/or modified under the
% conditions of the LaTeX Project Public License, either
% version 1.3c of this license or (at your option) any later
% version. This version of this license is in
%    http://www.latex-project.org/lppl/lppl-1-3c.txt
% and the latest version of this license is in
%    http://www.latex-project.org/lppl.txt
% and version 1.3 or later is part of all distributions of
% LaTeX version 2005/12/01 or later.
%
% This work has the LPPL maintenance status `maintained'.
% The Current Maintainer of this work is Haoqing Xu.
%
% Home Page of the Project: https://github.com/SuikaXhq/seu-bachelor-thesis-2019
%
% \fi
% \iffalse
%<*driver>
\ProvidesFile{seuthesis-2019.dtx}
%</driver>
%<cls>\NeedsTeXFormat{LaTeX2e}[2005/12/01]
%<cls>\ProvidesClass{seuthesis-2019}
%<cfg>\ProvidesFile{seuthesis-2019.cfg}
%<*cls>
  [2022.4.16 v0.1 LaTeX Template for SEU Bachelor Thesis ver2019]
%</cls>
%<*driver>
\documentclass[10pt,a4paper,oneside]{ltxdoc}
\usepackage{seuthesisdoc}
\EnableCrossrefs
\CodelineIndex
\RecordChanges
\begin{document}
  \DocInput{seuthesis-2019.dtx}
\end{document}
%</driver>
% \fi

% \CheckSum{0}
%
% \CharacterTable
%  {Upper-case    \A\B\C\D\E\F\G\H\I\J\K\L\M\N\O\P\Q\R\S\T\U\V\W\X\Y\Z
%   Lower-case    \a\b\c\d\e\f\g\h\i\j\k\l\m\n\o\p\q\r\s\t\u\v\w\x\y\z
%   Digits        \0\1\2\3\4\5\6\7\8\9
%   Exclamation   \!     Double quote  \"     Hash (number) \#
%   Dollar        \$     Percent       \%     Ampersand     \&
%   Acute accent  \'     Left paren    \(     Right paren   \)
%   Asterisk      \*     Plus          \+     Comma         \,
%   Minus         \-     Point         \.     Solidus       \/
%   Colon         \:     Semicolon     \;     Less than     \<
%   Equals        \=     Greater than  \>     Question mark \?
%   Commercial at \@     Left bracket  \[     Backslash     \\
%   Right bracket \]     Circumflex    \^     Underscore    \_
%   Grave accent  \`     Left brace    \{     Vertical bar  \|
%   Right brace   \}     Tilde         \~}
% 
% \StopEventually{\PrintIndex}

% \section{实现细节}
% \subsection{字段定义}
% 定义模版内常用字段,包括章节标题、固定项名称等。该部分保存于文档配置文件(*.cfg)。变量命名规则为seut@功能@类型@名称,功能包含预设字段(field)、变量(var)等,类型包括字符串(str)、数值(num)、布尔(bool)等,名称可能包含所属页面等辅助信息。
% 
% 定义常用变量。
%    \begin{macrocode}
%<*cfg>
\newcommand*{\seut@space}[1]{\hspace{#1\ccwd}}
%    \end{macrocode}
% 定义封面页字段。
%    \begin{macrocode}
\newcommand*{\seut@const@str@titlepage@logotitle}     {毕业设计(论文)报告}
\newcommand*{\seut@field@str@titlepage@title}         {题\seut@space{0.5}目:}
\newcommand*{\seut@field@str@titlepage@studentnum}    {学\seut@space{2}号:}
\newcommand*{\seut@field@str@titlepage@author}        {姓\seut@space{2}名:}
\newcommand*{\seut@field@str@titlepage@department}    {学\seut@space{2}院:}
\newcommand*{\seut@field@str@titlepage@major}         {专\seut@space{2}业:}
\newcommand*{\seut@field@str@titlepage@supervisor}    {指导教师:}
\newcommand*{\seut@field@str@titlepage@period}        {起止日期:}

\newif\ifseut@title@long\seut@title@longtrue
\newcommand*{\seut@var@str@titlepage@title}{}
\newcommand*{\seut@var@str@titlepage@title@up}{}
\newcommand*{\seut@var@str@titlepage@title@down}{}
\newcommand*{\seut@var@str@titlepage@studentnum}{}
\newcommand*{\seut@var@str@titlepage@author}{}
\newcommand*{\seut@var@str@titlepage@department}{}
\newcommand*{\seut@var@str@titlepage@major}{}
\newcommand*{\seut@var@str@titlepage@supervisor}{}
\newcommand*{\seut@var@str@titlepage@period}{}
%    \end{macrocode}
% 定义声明页字段。
%    \begin{macrocode}
\newcommand*{\seut@const@str@declare@origin@title}{东南大学毕业(设计)论文独创性声明}
\newcommand*{\seut@const@str@declare@origin@content}{%
  本人声明所呈交的毕业(设计)论文是我个人在导师指导下进行的研究工作及取得的研究成果。尽我所知,%
  除了文中特别加以标注和致谢的地方外,论文中不包含其他人已经发表或撰写过的研究成果,也不包含为获%
  得东南大学或其它教育机构的学位或证书而使用过的材料。与我一同工作的同志对本研究所做的任何贡献均%
  已在论文中作了明确的说明并表示了谢意。%
}
\newcommand*{\seut@field@str@declare@authorsign}{论文作者签名:}
\newcommand*{\seut@field@str@declare@date}{日期:}
\newcommand*{\seut@field@str@declare@supervisorsign}{导师签名:}
\newcommand*{\seut@const@str@declare@usage@title}{东南大学毕业(设计)论文使用授权声明}
\newcommand*{\seut@const@str@declare@usage@content}{%
  东南大学有权保留本人所送交毕业(设计)论文的复印件和电子文档,可以采用影印、缩印或其他复制手段%
  保存论文。本人电子文档的内容和纸质论文的内容相一致。除在保密期内的保密论文外,允许论文被查阅和%
  借阅,可以公布(包括刊登)论文的全部或部分内容。论文的公布(包括刊登)授权东南大学教务处办理。%
}
%    \end{macrocode}
% 定义字符串常量。
%    \begin{macrocode}
\newcommand*{\seut@const@str@headertext}{东南大学本科毕业设计(论文)}
%    \end{macrocode}
% 定义英文字体名称,包括衬线、无衬线、等宽字体。
%    \begin{macrocode}
\newcommand*{\seut@const@str@font@serif@win}        {Times New Roman}
\newcommand*{\seut@const@str@font@sans@win}         {Arial}
\newcommand*{\seut@const@str@font@mono@win}         {Courier New}
%    \end{macrocode}
% 定义中文字体名称,包括宋体、黑体、仿宋、楷体四种常用字体。
%    \begin{macrocode}
\newcommand*{\seut@const@str@font@songti@FZ}        {FZShuSong-Z01}
\newcommand*{\seut@const@str@font@heiti@FZ}         {FZHei-B01}
\newcommand*{\seut@const@str@font@fangsong@FZ}      {FZFangSong-Z02}
\newcommand*{\seut@const@str@font@kaiti@FZ}         {FZKai-Z03}

\newcommand*{\seut@const@str@font@songti@win}       {SimSun}
\newcommand*{\seut@const@str@font@heiti@win}        {SimHei}
\newcommand*{\seut@const@str@font@fangsong@win}     {FangSong}
\newcommand*{\seut@const@str@font@kaiti@win}        {KaiTi}
%    \end{macrocode}
% 定义字体选项布尔值,默认为Windows字体。
%    \begin{macrocode}
\newif\ifseut@font@win\seut@font@wintrue
\newif\ifseut@font@mac\seut@font@macfalse
%    \end{macrocode}
%</cfg>

% \subsection{启动前检查}
% 检查\XeLaTeX引擎。
%    \begin{macrocode}
%<*cls>
\RequirePackage{ifxetex}
\RequireXeTeX
%    \end{macrocode}

% \subsection{文档类选项处理}
% 导入配置文件。
%    \begin{macrocode}
\input{seuthesis-2019.cfg}
%    \end{macrocode}
% 处理字体设置。
%    \begin{macrocode}
\DeclareOption{winfonts}{%
  \seut@font@wintrue
  \seut@font@macfalse
}
\DeclareOption{macfonts}{%
  \seut@font@winfalse
  \seut@font@mactrue
}
%    \end{macrocode}
% 传递剩余选项给底层文档类。
%    \begin{macrocode}
\DeclareOption*{\PassOptionsToClass{\CurrentOption}{ctexrep}}
%    \end{macrocode}
% 结束选项处理。
%    \begin{macrocode}
\ProcessOptions\relax
%    \end{macrocode}

% \subsection{宏包}
% 调用底层文档类,此处指定默认字号为小四号。
%    \begin{macrocode}
\LoadClass[UTF8,zihao=-4]{ctexrep}
%    \end{macrocode}
% 导入页面设置宏包。
%    \begin{macrocode}
\RequirePackage{geometry}
\RequirePackage{titlesec}
\RequirePackage{setspace}
\RequirePackage{fancyhdr}
%    \end{macrocode}
% 导入图片、表格设置宏包。
%    \begin{macrocode}
\RequirePackage{graphicx}
\RequirePackage{caption}
\RequirePackage{subcaption}
\RequirePackage{float}
\RequirePackage{array}
\RequirePackage{longtable}
\RequirePackage{booktabs}
\RequirePackage{multirow}
%    \end{macrocode}
% 导入数学环境宏包。
%    \begin{macrocode}
\RequirePackage{amsmath}
\RequirePackage{amsthm}
\RequirePackage{amsfonts}
\RequirePackage{amssymb}
%    \end{macrocode}
% 导入引用设置宏包并添加hyperref设置。
%    \begin{macrocode}
\RequirePackage[hyphens]{url}
\RequirePackage{hyperref}
\hypersetup{
    unicode=false,
    hyperfootnotes=true,
    hyperindex=true,
    pageanchor=true,
    CJKbookmarks=true,
    bookmarksnumbered=true,
    bookmarksopen=true,
    bookmarksopenlevel=0,
    breaklinks=true,
    colorlinks=false,
    plainpages=false,
    pdfpagelabels,
    pdfborder=0 0 0,
}
\urlstyle{same}
\RequirePackage[sort&compress]{gbt7714}
\RequirePackage{hypernat}
%    \end{macrocode}
% 导入一些杂项宏包。其中部分宏包必须在前述宏包后导入。
%    \begin{macrocode}
\RequirePackage{ifthen}
\RequirePackage{tabularx}
\RequirePackage{calc}
\RequirePackage{etoolbox}
\RequirePackage{xeCJKfntef}
\RequirePackage{doc}
%    \end{macrocode}

% \subsection{页面布局设置}
% 页边距(2厘米)及装订线(左侧0.5厘米)设置。Word版模版中,上下边距实际上由页眉页脚定义,因此此处无需定义上下页边距。
%    \begin{macrocode}
\geometry{
  a4paper,
  left=2.5cm,
  right=2cm,
  includehead=true,
  includefoot=true,
}
%    \end{macrocode}
% 页眉页脚边距设置,其行距为22磅,页眉顶端距离1.5厘米,页脚底端距离1.75厘米。
%    \begin{macrocode}
\geometry{
  headheight=22pt,
  headsep=0pt,
  footskip=22pt,
  top=1.5cm,
  bottom=1.75cm,
}
%    \end{macrocode}

% \subsection{字体设置}
% 选择合适的中文字体名称变量。
%    \begin{macrocode}
\ifseut@font@win
\newcommand*{\seut@var@str@font@serif}      {\seut@const@str@font@serif@win}
\newcommand*{\seut@var@str@font@sans}       {\seut@const@str@font@sans@win}
\newcommand*{\seut@var@str@font@mono}       {\seut@const@str@font@mono@win}

\newcommand*{\seut@var@str@font@songti}     {\seut@const@str@font@songti@win}
\newcommand*{\seut@var@str@font@heiti}      {\seut@const@str@font@heiti@win}
\newcommand*{\seut@var@str@font@fangsong}   {\seut@const@str@font@fangsong@win}
\newcommand*{\seut@var@str@font@kaiti}      {\seut@const@str@font@kaiti@win}
\fi

\ifseut@font@mac
\newcommand*{\seut@var@str@font@serif}      {\seut@const@str@font@serif@mac}
\newcommand*{\seut@var@str@font@sans}       {\seut@const@str@font@sans@mac}
\newcommand*{\seut@var@str@font@mono}       {\seut@const@str@font@mono@mac}

\newcommand*{\seut@var@str@font@songti}     {\seut@const@str@font@songti@mac}
\newcommand*{\seut@var@str@font@heiti}      {\seut@const@str@font@heiti@mac}
\newcommand*{\seut@var@str@font@fangsong}   {\seut@const@str@font@fangsong@mac}
\newcommand*{\seut@var@str@font@kaiti}      {\seut@const@str@font@kaiti@mac}
\fi
%    \end{macrocode}
% 设置中文字体。启用伪粗体和伪斜体,设置主要中文字体为宋体、无衬线中文字体为黑体、等宽中文字体为仿宋。
%    \begin{macrocode}
\xeCJKsetup{
  AutoFakeBold=true,
  AutoFakeSlant=true,
}
\setCJKfamilyfont{zhsong}{\seut@var@str@font@songti}
\setCJKfamilyfont{zhhei}{\seut@var@str@font@heiti}
\setCJKfamilyfont{zhfs}{\seut@var@str@font@fangsong}
\setCJKfamilyfont{zhkai}{\seut@var@str@font@kaiti}
\setCJKmainfont{\seut@var@str@font@songti}
\setCJKsansfont{\seut@var@str@font@heiti}
\setCJKmonofont{\seut@var@str@font@fangsong}
%    \end{macrocode}
% 设置英文字体。
%    \begin{macrocode}
\setmainfont{\seut@var@str@font@serif}
\setsansfont{\seut@var@str@font@sans}
\setmonofont{\seut@var@str@font@mono}
%    \end{macrocode}

% \subsection{页眉页脚}
% 设置页眉页脚格式,包括无样式(plain)、前页样式(front)、正文样式(main)。
%    \begin{macrocode}
\fancypagestyle{plain}{%
  \fancyhf{}%
  \renewcommand{\headrulewidth}{0pt}%
  \renewcommand{\footrulewidth}{0pt}%
}
\fancypagestyle{front}{%
  \fancyhf{}%
  \renewcommand{\headrulewidth}{0pt}%
  \renewcommand{\footrulewidth}{0pt}%
  \fancyfoot[C]{\Roman{\thepage}}%
}
\fancypagestyle{main}{%
  \fancyhf{}%
  \renewcommand{\headrulewidth}{0.75pt}%
  \renewcommand{\footrulewidth}{0pt}%
  \fancyhead[C]{\seut@const@str@headertext}%
  \fancyfoot[C]{\arabic{\thepage}}%
}
\pagestyle{plain}
%    \end{macrocode}
% \subsection{文档结构}
% 设置关于毕业设计(论文)的文档结构指令。包括摘要目录部分(前页部分,frontmatter)以及正文部分(mainmatter)并指定页编号和页眉页脚样式。
%    \begin{macrocode}
\newcommand{\frontmatter}{%
  \clearpage
  \pagenumbering{Roman}
  \pagestyle{front}
}
\newcommand{\mainmatter}{%
  \clearpage
  \pagenumbering{arabic}
  \pagestyle{main}
}
%    \end{macrocode}
% \subsection{段落设置}
% 正文行间距设置。注意到Word中倍率行间距取决于文档行网络,而Word模版中行网络间距定为16.3pt,约为小四号字体(12pt)的1.35倍,其对应1.5倍行间距为24.45pt,即约2.04倍字体高度。\LaTeX中的基线距离(即Word行间距)为\cs{baselineskip}与\cs{linespread}的乘积,\cs{baselineskip}为字号的1.2倍,则需设置\cs{linespread}为1.7以匹配Word设置。声明页Word行间距为1.25倍(对应\cs{linespread}$=$1.4)。由于生成声明页需要切换行间距,此处定义为文档命令。
%    \begin{macrocode}
\newcommand*{\seut@control@mainlinespread}{\renewcommand*{\linespread}{1.7}}
\newcommand*{\seut@control@declarelinespread}{\renewcommand*{\linespread}{1.4}}
\seut@control@mainlinespread
%    \end{macrocode}

% \subsection{浮动体设置}

% \subsection{注释脚注}

% \subsection{数学环境}

% \subsection{封面页}
% 定义封面表单的填写命令,包括标题(上/下)、学号、姓名、专业、学院、导师、起止时间等。
%    \begin{macrocode}
\renewcommand{\title}[2]{%
  \renewcommand*{\seut@var@str@titlepage@title}{#1#2}
  \renewcommand*{\seut@var@str@titlepage@title@up}{#1}
  \renewcommand*{\seut@var@str@titlepage@title@down}{#2}}
\newcommand{\studentnum}[1]{\renewcommand*{\seut@var@str@titlepage@studentnum}{#1}}
\renewcommand{\author}[1]{\renewcommand*{\seut@var@str@titlepage@author}{#1}}
\newcommand{\department}[1]{\renewcommand*{\seut@var@str@titlepage@department}{#1}}
\newcommand{\major}[1]{\renewcommand*{\seut@var@str@titlepage@major}{#1}}
\newcommand{\supervisor}[1]{\renewcommand*{\seut@var@str@titlepage@supervisor}{#1}}
\newcommand{\period}[1]{\renewcommand*{\seut@var@str@titlepage@period}{#1}}
%    \end{macrocode}
% 定义表单项占位符实现,包括可能的内容和下划线。
%    \begin{macrocode}
\newcommand{\seut@control@placeholder}[2][\linewidth]{\CJKunderline{\makebox[#1]{#2}}}
\def\seutplaceholder{\@ifnextchar[\seut@control@placeholder\CJKunderline}
%    \end{macrocode}
% 制作标题页,首先定义一些PDF Meta-Data。
%    \begin{macrocode}
\renewcommand{\maketitle}{%
  \GetFileInfo{seuthesis-2019.cls}
  \hypersetup{
    pdfauthor={\seut@var@str@titlepage@author},
    pdfcreator={\fileversion}
  }
%    \end{macrocode}
% 定义封面页。首先空出44pt(来自于Word模版内空行)的距离,放置东南大学文字Logo以及总标题。
%    \begin{macrocode}
  \begin{titlepage}
    \vspace*{44pt}
    \begin{centering}
      \includegraphics[width=7.2cm]{resources/seu-text-logo}\\ \relax
      \vspace{34pt}
      {\CJKfamily{zhsong}\zihao{-0}\textbf{\seut@const@str@titlepage@logotitle}}\\ \relax
    \end{centering}
%    \end{macrocode}
% 空开46pt(Word内空行),放置论文题目。
%    \begin{macrocode}
    \vspace{46pt}
    {\centering
    \heiti\zihao{2}
    \renewcommand*{\arraystretch}{1.65}
    \begin{tabular}{p{2.5cm}c}
      \makebox[2.5cm][r]{\seut@field@str@titlepage@title}
      & \seutplaceholder[12.5cm]{\seut@var@str@titlepage@title@up} \\ \relax
      \ifcsempty{seut@var@str@titlepage@title@down}{}{
      & \seutplaceholder[12.5cm]{\seut@var@str@titlepage@title@down}}
    \end{tabular}
    \par % to make centering work
    }

%    \end{macrocode}
% 空开46pt(Word内空行),放置封面其他信息。
%    \begin{macrocode}
    \vspace{46pt}
    {\centering
    \zihao{-2}
    \renewcommand*{\arraystretch}{1.4}
    \begin{tabular}{p{3cm}c}
      \makebox[3cm][r]{\seut@field@str@titlepage@studentnum}
      & \seutplaceholder[8cm]{\seut@var@str@titlepage@studentnum} \\ \relax
      \makebox[3cm][r]{\seut@field@str@titlepage@author}
      & \seutplaceholder[8cm]{\seut@var@str@titlepage@author} \\ \relax
      \makebox[3cm][r]{\seut@field@str@titlepage@department}
      & \seutplaceholder[8cm]{\seut@var@str@titlepage@department} \\ \relax
      \makebox[3cm][r]{\seut@field@str@titlepage@major}
      & \seutplaceholder[8cm]{\seut@var@str@titlepage@major} \\ \relax
      \makebox[3cm][r]{\seut@field@str@titlepage@supervisor}
      & \seutplaceholder[8cm]{\seut@var@str@titlepage@supervisor} \\ \relax
      \makebox[3cm][r]{\seut@field@str@titlepage@period}
      & \seutplaceholder[8cm]{\seut@var@str@titlepage@period}
    \end{tabular}
    \par % to make centering work
    }
  \end{titlepage}
}
%    \end{macrocode}
% \subsection{声明页}

% \subsection{特殊章节}

%</cls>
% \Finale

% \iffalse
% TODO: style
%<*doc>
\ProvidesPackage{seuthesisdoc}
%</doc>
% \fi
\endinput
%    \end{macrocode}
% 处理字体设置。
%    \begin{macrocode}
\DeclareOption{winfonts}{%
  \seut@font@wintrue
  \seut@font@macfalse
}
\DeclareOption{macfonts}{%
  \seut@font@winfalse
  \seut@font@mactrue
}
%    \end{macrocode}
% 传递剩余选项给底层文档类。
%    \begin{macrocode}
\DeclareOption*{\PassOptionsToClass{\CurrentOption}{ctexrep}}
%    \end{macrocode}
% 结束选项处理。
%    \begin{macrocode}
\ProcessOptions\relax
%    \end{macrocode}

% \subsection{宏包}
% 调用底层文档类,此处指定默认字号为小四号。
%    \begin{macrocode}
\LoadClass[UTF8,zihao=-4]{ctexrep}
%    \end{macrocode}
% 导入页面设置宏包。
%    \begin{macrocode}
\RequirePackage{geometry}
\RequirePackage{titlesec}
\RequirePackage{setspace}
\RequirePackage{fancyhdr}
%    \end{macrocode}
% 导入图片、表格设置宏包。
%    \begin{macrocode}
\RequirePackage{graphicx}
\RequirePackage{caption}
\RequirePackage{subcaption}
\RequirePackage{float}
\RequirePackage{array}
\RequirePackage{longtable}
\RequirePackage{booktabs}
\RequirePackage{multirow}
%    \end{macrocode}
% 导入数学环境宏包。
%    \begin{macrocode}
\RequirePackage{amsmath}
\RequirePackage{amsthm}
\RequirePackage{amsfonts}
\RequirePackage{amssymb}
%    \end{macrocode}
% 导入引用设置宏包并添加hyperref设置。
%    \begin{macrocode}
\RequirePackage[hyphens]{url}
\RequirePackage{hyperref}
\hypersetup{
    unicode=false,
    hyperfootnotes=true,
    hyperindex=true,
    pageanchor=true,
    CJKbookmarks=true,
    bookmarksnumbered=true,
    bookmarksopen=true,
    bookmarksopenlevel=0,
    breaklinks=true,
    colorlinks=false,
    plainpages=false,
    pdfpagelabels,
    pdfborder=0 0 0,
}
\urlstyle{same}
\RequirePackage[sort&compress]{gbt7714}
\RequirePackage{hypernat}
%    \end{macrocode}
% 导入一些杂项宏包。其中部分宏包必须在前述宏包后导入。
%    \begin{macrocode}
\RequirePackage{ifthen}
\RequirePackage{tabularx}
\RequirePackage{calc}
\RequirePackage{etoolbox}
\RequirePackage{xeCJKfntef}
\RequirePackage{doc}
%    \end{macrocode}

% \subsection{页面布局设置}
% 页边距(2厘米)及装订线(左侧0.5厘米)设置。Word版模版中,上下边距实际上由页眉页脚定义,因此此处无需定义上下页边距。
%    \begin{macrocode}
\geometry{
  a4paper,
  left=2.5cm,
  right=2cm,
  includehead=true,
  includefoot=true,
}
%    \end{macrocode}
% 页眉页脚边距设置,其行距为22磅,页眉顶端距离1.5厘米,页脚底端距离1.75厘米。
%    \begin{macrocode}
\geometry{
  headheight=22pt,
  headsep=0pt,
  footskip=22pt,
  top=1.5cm,
  bottom=1.75cm,
}
%    \end{macrocode}

% \subsection{字体设置}
% 选择合适的中文字体名称变量。
%    \begin{macrocode}
\ifseut@font@win
\newcommand*{\seut@var@str@font@serif}      {\seut@const@str@font@serif@win}
\newcommand*{\seut@var@str@font@sans}       {\seut@const@str@font@sans@win}
\newcommand*{\seut@var@str@font@mono}       {\seut@const@str@font@mono@win}

\newcommand*{\seut@var@str@font@songti}     {\seut@const@str@font@songti@win}
\newcommand*{\seut@var@str@font@heiti}      {\seut@const@str@font@heiti@win}
\newcommand*{\seut@var@str@font@fangsong}   {\seut@const@str@font@fangsong@win}
\newcommand*{\seut@var@str@font@kaiti}      {\seut@const@str@font@kaiti@win}
\fi

\ifseut@font@mac
\newcommand*{\seut@var@str@font@serif}      {\seut@const@str@font@serif@mac}
\newcommand*{\seut@var@str@font@sans}       {\seut@const@str@font@sans@mac}
\newcommand*{\seut@var@str@font@mono}       {\seut@const@str@font@mono@mac}

\newcommand*{\seut@var@str@font@songti}     {\seut@const@str@font@songti@mac}
\newcommand*{\seut@var@str@font@heiti}      {\seut@const@str@font@heiti@mac}
\newcommand*{\seut@var@str@font@fangsong}   {\seut@const@str@font@fangsong@mac}
\newcommand*{\seut@var@str@font@kaiti}      {\seut@const@str@font@kaiti@mac}
\fi
%    \end{macrocode}
% 设置中文字体。启用伪粗体和伪斜体,设置主要中文字体为宋体、无衬线中文字体为黑体、等宽中文字体为仿宋。
%    \begin{macrocode}
\xeCJKsetup{
  AutoFakeBold=true,
  AutoFakeSlant=true,
}
\setCJKfamilyfont{zhsong}{\seut@var@str@font@songti}
\setCJKfamilyfont{zhhei}{\seut@var@str@font@heiti}
\setCJKfamilyfont{zhfs}{\seut@var@str@font@fangsong}
\setCJKfamilyfont{zhkai}{\seut@var@str@font@kaiti}
\setCJKmainfont{\seut@var@str@font@songti}
\setCJKsansfont{\seut@var@str@font@heiti}
\setCJKmonofont{\seut@var@str@font@fangsong}
%    \end{macrocode}
% 设置英文字体。
%    \begin{macrocode}
\setmainfont{\seut@var@str@font@serif}
\setsansfont{\seut@var@str@font@sans}
\setmonofont{\seut@var@str@font@mono}
%    \end{macrocode}

% \subsection{页眉页脚}
% 设置页眉页脚格式,包括无样式(plain)、前页样式(front)、正文样式(main)。
%    \begin{macrocode}
\fancypagestyle{plain}{%
  \fancyhf{}%
  \renewcommand{\headrulewidth}{0pt}%
  \renewcommand{\footrulewidth}{0pt}%
}
\fancypagestyle{front}{%
  \fancyhf{}%
  \renewcommand{\headrulewidth}{0pt}%
  \renewcommand{\footrulewidth}{0pt}%
  \fancyfoot[C]{\Roman{\thepage}}%
}
\fancypagestyle{main}{%
  \fancyhf{}%
  \renewcommand{\headrulewidth}{0.75pt}%
  \renewcommand{\footrulewidth}{0pt}%
  \fancyhead[C]{\seut@const@str@headertext}%
  \fancyfoot[C]{\arabic{\thepage}}%
}
\pagestyle{plain}
%    \end{macrocode}
% \subsection{文档结构}
% 设置关于毕业设计(论文)的文档结构指令。包括摘要目录部分(前页部分,frontmatter)以及正文部分(mainmatter)并指定页编号和页眉页脚样式。
%    \begin{macrocode}
\newcommand{\frontmatter}{%
  \clearpage
  \pagenumbering{Roman}
  \pagestyle{front}
}
\newcommand{\mainmatter}{%
  \clearpage
  \pagenumbering{arabic}
  \pagestyle{main}
}
%    \end{macrocode}
% \subsection{段落设置}
% 正文行间距设置。注意到Word中倍率行间距取决于文档行网络,而Word模版中行网络间距定为16.3pt,约为小四号字体(12pt)的1.35倍,其对应1.5倍行间距为24.45pt,即约2.04倍字体高度。\LaTeX中的基线距离(即Word行间距)为\cs{baselineskip}与\cs{linespread}的乘积,\cs{baselineskip}为字号的1.2倍,则需设置\cs{linespread}为1.7以匹配Word设置。声明页Word行间距为1.25倍(对应\cs{linespread}$=$1.4)。由于生成声明页需要切换行间距,此处定义为文档命令。
%    \begin{macrocode}
\newcommand*{\seut@control@mainlinespread}{\renewcommand*{\linespread}{1.7}}
\newcommand*{\seut@control@declarelinespread}{\renewcommand*{\linespread}{1.4}}
\seut@control@mainlinespread
%    \end{macrocode}

% \subsection{浮动体设置}

% \subsection{注释脚注}

% \subsection{数学环境}

% \subsection{封面页}
% 定义封面表单的填写命令,包括标题(上/下)、学号、姓名、专业、学院、导师、起止时间等。
%    \begin{macrocode}
\renewcommand{\title}[2]{%
  \renewcommand*{\seut@var@str@titlepage@title}{#1#2}
  \renewcommand*{\seut@var@str@titlepage@title@up}{#1}
  \renewcommand*{\seut@var@str@titlepage@title@down}{#2}}
\newcommand{\studentnum}[1]{\renewcommand*{\seut@var@str@titlepage@studentnum}{#1}}
\renewcommand{\author}[1]{\renewcommand*{\seut@var@str@titlepage@author}{#1}}
\newcommand{\department}[1]{\renewcommand*{\seut@var@str@titlepage@department}{#1}}
\newcommand{\major}[1]{\renewcommand*{\seut@var@str@titlepage@major}{#1}}
\newcommand{\supervisor}[1]{\renewcommand*{\seut@var@str@titlepage@supervisor}{#1}}
\newcommand{\period}[1]{\renewcommand*{\seut@var@str@titlepage@period}{#1}}
%    \end{macrocode}
% 定义表单项占位符实现,包括可能的内容和下划线。
%    \begin{macrocode}
\newcommand{\seut@control@placeholder}[2][\linewidth]{\CJKunderline{\makebox[#1]{#2}}}
\def\seutplaceholder{\@ifnextchar[\seut@control@placeholder\CJKunderline}
%    \end{macrocode}
% 制作标题页,首先定义一些PDF Meta-Data。
%    \begin{macrocode}
\renewcommand{\maketitle}{%
  \GetFileInfo{seuthesis-2019.cls}
  \hypersetup{
    pdfauthor={\seut@var@str@titlepage@author},
    pdfcreator={\fileversion}
  }
%    \end{macrocode}
% 定义封面页。首先空出44pt(来自于Word模版内空行)的距离,放置东南大学文字Logo以及总标题。
%    \begin{macrocode}
  \begin{titlepage}
    \vspace*{44pt}
    \begin{centering}
      \includegraphics[width=7.2cm]{resources/seu-text-logo}\\ \relax
      \vspace{34pt}
      {\CJKfamily{zhsong}\zihao{-0}\textbf{\seut@const@str@titlepage@logotitle}}\\ \relax
    \end{centering}
%    \end{macrocode}
% 空开46pt(Word内空行),放置论文题目。
%    \begin{macrocode}
    \vspace{46pt}
    {\centering
    \heiti\zihao{2}
    \renewcommand*{\arraystretch}{1.65}
    \begin{tabular}{p{2.5cm}c}
      \makebox[2.5cm][r]{\seut@field@str@titlepage@title}
      & \seutplaceholder[12.5cm]{\seut@var@str@titlepage@title@up} \\ \relax
      \ifcsempty{seut@var@str@titlepage@title@down}{}{
      & \seutplaceholder[12.5cm]{\seut@var@str@titlepage@title@down}}
    \end{tabular}
    \par % to make centering work
    }

%    \end{macrocode}
% 空开46pt(Word内空行),放置封面其他信息。
%    \begin{macrocode}
    \vspace{46pt}
    {\centering
    \zihao{-2}
    \renewcommand*{\arraystretch}{1.4}
    \begin{tabular}{p{3cm}c}
      \makebox[3cm][r]{\seut@field@str@titlepage@studentnum}
      & \seutplaceholder[8cm]{\seut@var@str@titlepage@studentnum} \\ \relax
      \makebox[3cm][r]{\seut@field@str@titlepage@author}
      & \seutplaceholder[8cm]{\seut@var@str@titlepage@author} \\ \relax
      \makebox[3cm][r]{\seut@field@str@titlepage@department}
      & \seutplaceholder[8cm]{\seut@var@str@titlepage@department} \\ \relax
      \makebox[3cm][r]{\seut@field@str@titlepage@major}
      & \seutplaceholder[8cm]{\seut@var@str@titlepage@major} \\ \relax
      \makebox[3cm][r]{\seut@field@str@titlepage@supervisor}
      & \seutplaceholder[8cm]{\seut@var@str@titlepage@supervisor} \\ \relax
      \makebox[3cm][r]{\seut@field@str@titlepage@period}
      & \seutplaceholder[8cm]{\seut@var@str@titlepage@period}
    \end{tabular}
    \par % to make centering work
    }
  \end{titlepage}
}
%    \end{macrocode}
% \subsection{声明页}

% \subsection{特殊章节}

%</cls>
% \Finale

% \iffalse
% TODO: style
%<*doc>
\ProvidesPackage{seuthesisdoc}
%</doc>
% \fi
\endinput
%    \end{macrocode}
% 处理字体设置。
%    \begin{macrocode}
\DeclareOption{winfonts}{%
  \seut@font@wintrue
  \seut@font@macfalse
}
\DeclareOption{macfonts}{%
  \seut@font@winfalse
  \seut@font@mactrue
}
%    \end{macrocode}
% 传递剩余选项给底层文档类。
%    \begin{macrocode}
\DeclareOption*{\PassOptionsToClass{\CurrentOption}{ctexrep}}
%    \end{macrocode}
% 结束选项处理。
%    \begin{macrocode}
\ProcessOptions\relax
%    \end{macrocode}

% \subsection{宏包}
% 调用底层文档类,此处指定默认字号为小四号。
%    \begin{macrocode}
\LoadClass[UTF8,zihao=-4]{ctexrep}
%    \end{macrocode}
% 导入页面设置宏包。
%    \begin{macrocode}
\RequirePackage{geometry}
\RequirePackage{titlesec}
\RequirePackage{setspace}
\RequirePackage{fancyhdr}
%    \end{macrocode}
% 导入图片、表格设置宏包。
%    \begin{macrocode}
\RequirePackage{graphicx}
\RequirePackage{caption}
\RequirePackage{subcaption}
\RequirePackage{float}
\RequirePackage{array}
\RequirePackage{longtable}
\RequirePackage{booktabs}
\RequirePackage{multirow}
%    \end{macrocode}
% 导入数学环境宏包。
%    \begin{macrocode}
\RequirePackage{amsmath}
\RequirePackage{amsthm}
\RequirePackage{amsfonts}
\RequirePackage{amssymb}
%    \end{macrocode}
% 导入引用设置宏包并添加hyperref设置。
%    \begin{macrocode}
\RequirePackage[hyphens]{url}
\RequirePackage{hyperref}
\hypersetup{
    unicode=false,
    hyperfootnotes=true,
    hyperindex=true,
    pageanchor=true,
    CJKbookmarks=true,
    bookmarksnumbered=true,
    bookmarksopen=true,
    bookmarksopenlevel=0,
    breaklinks=true,
    colorlinks=false,
    plainpages=false,
    pdfpagelabels,
    pdfborder=0 0 0,
}
\urlstyle{same}
\RequirePackage[sort&compress]{gbt7714}
\RequirePackage{hypernat}
%    \end{macrocode}
% 导入一些杂项宏包。其中部分宏包必须在前述宏包后导入。
%    \begin{macrocode}
\RequirePackage{ifthen}
\RequirePackage{tabularx}
\RequirePackage{calc}
\RequirePackage{etoolbox}
\RequirePackage{xeCJKfntef}
\RequirePackage{doc}
%    \end{macrocode}

% \subsection{页面布局设置}
% 页边距(2厘米)及装订线(左侧0.5厘米)设置。Word版模版中,上下边距实际上由页眉页脚定义,因此此处无需定义上下页边距。
%    \begin{macrocode}
\geometry{
  a4paper,
  left=2.5cm,
  right=2cm,
  includehead=true,
  includefoot=true,
}
%    \end{macrocode}
% 页眉页脚边距设置,其行距为22磅,页眉顶端距离1.5厘米,页脚底端距离1.75厘米。
%    \begin{macrocode}
\geometry{
  headheight=22pt,
  headsep=0pt,
  footskip=22pt,
  top=1.5cm,
  bottom=1.75cm,
}
%    \end{macrocode}

% \subsection{字体设置}
% 选择合适的中文字体名称变量。
%    \begin{macrocode}
\ifseut@font@win
\newcommand*{\seut@var@str@font@serif}      {\seut@const@str@font@serif@win}
\newcommand*{\seut@var@str@font@sans}       {\seut@const@str@font@sans@win}
\newcommand*{\seut@var@str@font@mono}       {\seut@const@str@font@mono@win}

\newcommand*{\seut@var@str@font@songti}     {\seut@const@str@font@songti@win}
\newcommand*{\seut@var@str@font@heiti}      {\seut@const@str@font@heiti@win}
\newcommand*{\seut@var@str@font@fangsong}   {\seut@const@str@font@fangsong@win}
\newcommand*{\seut@var@str@font@kaiti}      {\seut@const@str@font@kaiti@win}
\fi

\ifseut@font@mac
\newcommand*{\seut@var@str@font@serif}      {\seut@const@str@font@serif@mac}
\newcommand*{\seut@var@str@font@sans}       {\seut@const@str@font@sans@mac}
\newcommand*{\seut@var@str@font@mono}       {\seut@const@str@font@mono@mac}

\newcommand*{\seut@var@str@font@songti}     {\seut@const@str@font@songti@mac}
\newcommand*{\seut@var@str@font@heiti}      {\seut@const@str@font@heiti@mac}
\newcommand*{\seut@var@str@font@fangsong}   {\seut@const@str@font@fangsong@mac}
\newcommand*{\seut@var@str@font@kaiti}      {\seut@const@str@font@kaiti@mac}
\fi
%    \end{macrocode}
% 设置中文字体。启用伪粗体和伪斜体,设置主要中文字体为宋体、无衬线中文字体为黑体、等宽中文字体为仿宋。
%    \begin{macrocode}
\xeCJKsetup{
  AutoFakeBold=true,
  AutoFakeSlant=true,
}
\setCJKfamilyfont{zhsong}{\seut@var@str@font@songti}
\setCJKfamilyfont{zhhei}{\seut@var@str@font@heiti}
\setCJKfamilyfont{zhfs}{\seut@var@str@font@fangsong}
\setCJKfamilyfont{zhkai}{\seut@var@str@font@kaiti}
\setCJKmainfont{\seut@var@str@font@songti}
\setCJKsansfont{\seut@var@str@font@heiti}
\setCJKmonofont{\seut@var@str@font@fangsong}
%    \end{macrocode}
% 设置英文字体。
%    \begin{macrocode}
\setmainfont{\seut@var@str@font@serif}
\setsansfont{\seut@var@str@font@sans}
\setmonofont{\seut@var@str@font@mono}
%    \end{macrocode}

% \subsection{页眉页脚}
% 设置页眉页脚格式,包括无样式(plain)、前页样式(front)、正文样式(main)。
%    \begin{macrocode}
\fancypagestyle{plain}{%
  \fancyhf{}%
  \renewcommand{\headrulewidth}{0pt}%
  \renewcommand{\footrulewidth}{0pt}%
}
\fancypagestyle{front}{%
  \fancyhf{}%
  \renewcommand{\headrulewidth}{0pt}%
  \renewcommand{\footrulewidth}{0pt}%
  \fancyfoot[C]{\Roman{\thepage}}%
}
\fancypagestyle{main}{%
  \fancyhf{}%
  \renewcommand{\headrulewidth}{0.75pt}%
  \renewcommand{\footrulewidth}{0pt}%
  \fancyhead[C]{\seut@const@str@headertext}%
  \fancyfoot[C]{\arabic{\thepage}}%
}
\pagestyle{plain}
%    \end{macrocode}
% \subsection{文档结构}
% 设置关于毕业设计(论文)的文档结构指令。包括摘要目录部分(前页部分,frontmatter)以及正文部分(mainmatter)并指定页编号和页眉页脚样式。
%    \begin{macrocode}
\newcommand{\frontmatter}{%
  \clearpage
  \pagenumbering{Roman}
  \pagestyle{front}
}
\newcommand{\mainmatter}{%
  \clearpage
  \pagenumbering{arabic}
  \pagestyle{main}
}
%    \end{macrocode}
% \subsection{段落设置}
% 正文行间距设置。注意到Word中倍率行间距取决于文档行网络,而Word模版中行网络间距定为16.3pt,约为小四号字体(12pt)的1.35倍,其对应1.5倍行间距为24.45pt,即约2.04倍字体高度。\LaTeX中的基线距离(即Word行间距)为\cs{baselineskip}与\cs{linespread}的乘积,\cs{baselineskip}为字号的1.2倍,则需设置\cs{linespread}为1.7以匹配Word设置。声明页Word行间距为1.25倍(对应\cs{linespread}$=$1.4)。由于生成声明页需要切换行间距,此处定义为文档命令。
%    \begin{macrocode}
\newcommand*{\seut@control@mainlinespread}{\renewcommand*{\linespread}{1.7}}
\newcommand*{\seut@control@declarelinespread}{\renewcommand*{\linespread}{1.4}}
\seut@control@mainlinespread
%    \end{macrocode}

% \subsection{浮动体设置}

% \subsection{注释脚注}

% \subsection{数学环境}

% \subsection{封面页}
% 定义封面表单的填写命令,包括标题(上/下)、学号、姓名、专业、学院、导师、起止时间等。
%    \begin{macrocode}
\renewcommand{\title}[2]{%
  \renewcommand*{\seut@var@str@titlepage@title}{#1#2}
  \renewcommand*{\seut@var@str@titlepage@title@up}{#1}
  \renewcommand*{\seut@var@str@titlepage@title@down}{#2}}
\newcommand{\studentnum}[1]{\renewcommand*{\seut@var@str@titlepage@studentnum}{#1}}
\renewcommand{\author}[1]{\renewcommand*{\seut@var@str@titlepage@author}{#1}}
\newcommand{\department}[1]{\renewcommand*{\seut@var@str@titlepage@department}{#1}}
\newcommand{\major}[1]{\renewcommand*{\seut@var@str@titlepage@major}{#1}}
\newcommand{\supervisor}[1]{\renewcommand*{\seut@var@str@titlepage@supervisor}{#1}}
\newcommand{\period}[1]{\renewcommand*{\seut@var@str@titlepage@period}{#1}}
%    \end{macrocode}
% 定义表单项占位符实现,包括可能的内容和下划线。
%    \begin{macrocode}
\newcommand{\seut@control@placeholder}[2][\linewidth]{\CJKunderline{\makebox[#1]{#2}}}
\def\seutplaceholder{\@ifnextchar[\seut@control@placeholder\CJKunderline}
%    \end{macrocode}
% 制作标题页,首先定义一些PDF Meta-Data。
%    \begin{macrocode}
\renewcommand{\maketitle}{%
  \GetFileInfo{seuthesis-2019.cls}
  \hypersetup{
    pdfauthor={\seut@var@str@titlepage@author},
    pdfcreator={\fileversion}
  }
%    \end{macrocode}
% 定义封面页。首先空出44pt(来自于Word模版内空行)的距离,放置东南大学文字Logo以及总标题。
%    \begin{macrocode}
  \begin{titlepage}
    \vspace*{44pt}
    \begin{centering}
      \includegraphics[width=7.2cm]{resources/seu-text-logo}\\ \relax
      \vspace{34pt}
      {\CJKfamily{zhsong}\zihao{-0}\textbf{\seut@const@str@titlepage@logotitle}}\\ \relax
    \end{centering}
%    \end{macrocode}
% 空开46pt(Word内空行),放置论文题目。
%    \begin{macrocode}
    \vspace{46pt}
    {\centering
    \heiti\zihao{2}
    \renewcommand*{\arraystretch}{1.65}
    \begin{tabular}{p{2.5cm}c}
      \makebox[2.5cm][r]{\seut@field@str@titlepage@title}
      & \seutplaceholder[12.5cm]{\seut@var@str@titlepage@title@up} \\ \relax
      \ifcsempty{seut@var@str@titlepage@title@down}{}{
      & \seutplaceholder[12.5cm]{\seut@var@str@titlepage@title@down}}
    \end{tabular}
    \par % to make centering work
    }

%    \end{macrocode}
% 空开46pt(Word内空行),放置封面其他信息。
%    \begin{macrocode}
    \vspace{46pt}
    {\centering
    \zihao{-2}
    \renewcommand*{\arraystretch}{1.4}
    \begin{tabular}{p{3cm}c}
      \makebox[3cm][r]{\seut@field@str@titlepage@studentnum}
      & \seutplaceholder[8cm]{\seut@var@str@titlepage@studentnum} \\ \relax
      \makebox[3cm][r]{\seut@field@str@titlepage@author}
      & \seutplaceholder[8cm]{\seut@var@str@titlepage@author} \\ \relax
      \makebox[3cm][r]{\seut@field@str@titlepage@department}
      & \seutplaceholder[8cm]{\seut@var@str@titlepage@department} \\ \relax
      \makebox[3cm][r]{\seut@field@str@titlepage@major}
      & \seutplaceholder[8cm]{\seut@var@str@titlepage@major} \\ \relax
      \makebox[3cm][r]{\seut@field@str@titlepage@supervisor}
      & \seutplaceholder[8cm]{\seut@var@str@titlepage@supervisor} \\ \relax
      \makebox[3cm][r]{\seut@field@str@titlepage@period}
      & \seutplaceholder[8cm]{\seut@var@str@titlepage@period}
    \end{tabular}
    \par % to make centering work
    }
  \end{titlepage}
}
%    \end{macrocode}
% \subsection{声明页}

% \subsection{特殊章节}

%</cls>
% \Finale

% \iffalse
% TODO: style
%<*doc>
\ProvidesPackage{seuthesisdoc}
%</doc>
% \fi
\endinput
%    \end{macrocode}
% 处理字体设置。
%    \begin{macrocode}
\DeclareOption{winfonts}{%
  \seut@font@wintrue
  \seut@font@macfalse
}
\DeclareOption{macfonts}{%
  \seut@font@winfalse
  \seut@font@mactrue
}
%    \end{macrocode}
% 传递剩余选项给底层文档类。
%    \begin{macrocode}
\DeclareOption*{\PassOptionsToClass{\CurrentOption}{ctexrep}}
%    \end{macrocode}
% 结束选项处理。
%    \begin{macrocode}
\ProcessOptions\relax
%    \end{macrocode}

% \subsection{宏包}
% 调用底层文档类,此处指定默认字号为小四号。
%    \begin{macrocode}
\LoadClass[UTF8,zihao=-4]{ctexrep}
%    \end{macrocode}
% 导入页面设置宏包。
%    \begin{macrocode}
\RequirePackage{geometry}
\RequirePackage{titlesec}
\RequirePackage{setspace}
\RequirePackage{fancyhdr}
%    \end{macrocode}
% 导入图片、表格设置宏包。
%    \begin{macrocode}
\RequirePackage{graphicx}
\RequirePackage{caption}
\RequirePackage{subcaption}
\RequirePackage{float}
\RequirePackage{array}
\RequirePackage{longtable}
\RequirePackage{booktabs}
\RequirePackage{multirow}
%    \end{macrocode}
% 导入数学环境宏包。
%    \begin{macrocode}
\RequirePackage{amsmath}
\RequirePackage{amsthm}
\RequirePackage{amsfonts}
\RequirePackage{amssymb}
%    \end{macrocode}
% 导入引用设置宏包并添加hyperref设置。
%    \begin{macrocode}
\RequirePackage[hyphens]{url}
\RequirePackage{hyperref}
\hypersetup{
    unicode=false,
    hyperfootnotes=true,
    hyperindex=true,
    pageanchor=true,
    CJKbookmarks=true,
    bookmarksnumbered=true,
    bookmarksopen=true,
    bookmarksopenlevel=0,
    breaklinks=true,
    colorlinks=false,
    plainpages=false,
    pdfpagelabels,
    pdfborder=0 0 0,
}
\urlstyle{same}
\RequirePackage[sort&compress]{gbt7714}
\RequirePackage{hypernat}
%    \end{macrocode}
% 导入一些杂项宏包。其中部分宏包必须在前述宏包后导入。
%    \begin{macrocode}
\RequirePackage{ifthen}
\RequirePackage{tabularx}
\RequirePackage{calc}
\RequirePackage{etoolbox}
\RequirePackage{xeCJKfntef}
\RequirePackage{doc}
%    \end{macrocode}

% \subsection{页面布局设置}
% 页边距(2厘米)及装订线(左侧0.5厘米)设置。Word版模版中,上下边距实际上由页眉页脚定义,因此此处无需定义上下页边距。
%    \begin{macrocode}
\geometry{
  a4paper,
  left=2.5cm,
  right=2cm,
  includehead=true,
  includefoot=true,
}
%    \end{macrocode}
% 页眉页脚边距设置,其行距为22磅,页眉顶端距离1.5厘米,页脚底端距离1.75厘米。
%    \begin{macrocode}
\geometry{
  headheight=22pt,
  headsep=0pt,
  footskip=22pt,
  top=1.5cm,
  bottom=1.75cm,
}
%    \end{macrocode}

% \subsection{字体设置}
% 选择合适的中文字体名称变量。
%    \begin{macrocode}
\ifseut@font@win
\newcommand*{\seut@var@str@font@serif}      {\seut@const@str@font@serif@win}
\newcommand*{\seut@var@str@font@sans}       {\seut@const@str@font@sans@win}
\newcommand*{\seut@var@str@font@mono}       {\seut@const@str@font@mono@win}

\newcommand*{\seut@var@str@font@songti}     {\seut@const@str@font@songti@win}
\newcommand*{\seut@var@str@font@heiti}      {\seut@const@str@font@heiti@win}
\newcommand*{\seut@var@str@font@fangsong}   {\seut@const@str@font@fangsong@win}
\newcommand*{\seut@var@str@font@kaiti}      {\seut@const@str@font@kaiti@win}
\fi

\ifseut@font@mac
\newcommand*{\seut@var@str@font@serif}      {\seut@const@str@font@serif@mac}
\newcommand*{\seut@var@str@font@sans}       {\seut@const@str@font@sans@mac}
\newcommand*{\seut@var@str@font@mono}       {\seut@const@str@font@mono@mac}

\newcommand*{\seut@var@str@font@songti}     {\seut@const@str@font@songti@mac}
\newcommand*{\seut@var@str@font@heiti}      {\seut@const@str@font@heiti@mac}
\newcommand*{\seut@var@str@font@fangsong}   {\seut@const@str@font@fangsong@mac}
\newcommand*{\seut@var@str@font@kaiti}      {\seut@const@str@font@kaiti@mac}
\fi
%    \end{macrocode}
% 设置中文字体。启用伪粗体和伪斜体,设置主要中文字体为宋体、无衬线中文字体为黑体、等宽中文字体为仿宋。
%    \begin{macrocode}
\xeCJKsetup{
  AutoFakeBold=true,
  AutoFakeSlant=true,
}
\setCJKfamilyfont{zhsong}{\seut@var@str@font@songti}
\setCJKfamilyfont{zhhei}{\seut@var@str@font@heiti}
\setCJKfamilyfont{zhfs}{\seut@var@str@font@fangsong}
\setCJKfamilyfont{zhkai}{\seut@var@str@font@kaiti}
\setCJKmainfont{\seut@var@str@font@songti}
\setCJKsansfont{\seut@var@str@font@heiti}
\setCJKmonofont{\seut@var@str@font@fangsong}
%    \end{macrocode}
% 设置英文字体。
%    \begin{macrocode}
\setmainfont{\seut@var@str@font@serif}
\setsansfont{\seut@var@str@font@sans}
\setmonofont{\seut@var@str@font@mono}
%    \end{macrocode}

% \subsection{页眉页脚}
% 设置页眉页脚格式,包括无样式(plain)、前页样式(front)、正文样式(main)。
%    \begin{macrocode}
\fancypagestyle{plain}{%
  \fancyhf{}%
  \renewcommand{\headrulewidth}{0pt}%
  \renewcommand{\footrulewidth}{0pt}%
}
\fancypagestyle{front}{%
  \fancyhf{}%
  \renewcommand{\headrulewidth}{0pt}%
  \renewcommand{\footrulewidth}{0pt}%
  \fancyfoot[C]{\Roman{\thepage}}%
}
\fancypagestyle{main}{%
  \fancyhf{}%
  \renewcommand{\headrulewidth}{0.75pt}%
  \renewcommand{\footrulewidth}{0pt}%
  \fancyhead[C]{\seut@const@str@headertext}%
  \fancyfoot[C]{\arabic{\thepage}}%
}
\pagestyle{plain}
%    \end{macrocode}
% \subsection{文档结构}
% 设置关于毕业设计(论文)的文档结构指令。包括摘要目录部分(前页部分,frontmatter)以及正文部分(mainmatter)并指定页编号和页眉页脚样式。
%    \begin{macrocode}
\newcommand{\frontmatter}{%
  \clearpage
  \pagenumbering{Roman}
  \pagestyle{front}
}
\newcommand{\mainmatter}{%
  \clearpage
  \pagenumbering{arabic}
  \pagestyle{main}
}
%    \end{macrocode}
% \subsection{段落设置}
% 正文行间距设置。注意到Word中倍率行间距取决于文档行网络,而Word模版中行网络间距定为16.3pt,约为小四号字体(12pt)的1.35倍,其对应1.5倍行间距为24.45pt,即约2.04倍字体高度。\LaTeX中的基线距离(即Word行间距)为\cs{baselineskip}与\cs{linespread}的乘积,\cs{baselineskip}为字号的1.2倍,则需设置\cs{linespread}为1.7以匹配Word设置。声明页Word行间距为1.25倍(对应\cs{linespread}$=$1.4)。由于生成声明页需要切换行间距,此处定义为文档命令。
%    \begin{macrocode}
\newcommand*{\seut@control@mainlinespread}{\renewcommand*{\linespread}{1.7}}
\newcommand*{\seut@control@declarelinespread}{\renewcommand*{\linespread}{1.4}}
\seut@control@mainlinespread
%    \end{macrocode}

% \subsection{浮动体设置}

% \subsection{注释脚注}

% \subsection{数学环境}

% \subsection{封面页}
% 定义封面表单的填写命令,包括标题(上/下)、学号、姓名、专业、学院、导师、起止时间等。
%    \begin{macrocode}
\renewcommand{\title}[2]{%
  \renewcommand*{\seut@var@str@titlepage@title}{#1#2}
  \renewcommand*{\seut@var@str@titlepage@title@up}{#1}
  \renewcommand*{\seut@var@str@titlepage@title@down}{#2}}
\newcommand{\studentnum}[1]{\renewcommand*{\seut@var@str@titlepage@studentnum}{#1}}
\renewcommand{\author}[1]{\renewcommand*{\seut@var@str@titlepage@author}{#1}}
\newcommand{\department}[1]{\renewcommand*{\seut@var@str@titlepage@department}{#1}}
\newcommand{\major}[1]{\renewcommand*{\seut@var@str@titlepage@major}{#1}}
\newcommand{\supervisor}[1]{\renewcommand*{\seut@var@str@titlepage@supervisor}{#1}}
\newcommand{\period}[1]{\renewcommand*{\seut@var@str@titlepage@period}{#1}}
%    \end{macrocode}
% 定义表单项占位符实现,包括可能的内容和下划线。
%    \begin{macrocode}
\newcommand{\seut@control@placeholder}[2][\linewidth]{\CJKunderline{\makebox[#1]{#2}}}
\def\seutplaceholder{\@ifnextchar[\seut@control@placeholder\CJKunderline}
%    \end{macrocode}
% 制作标题页,首先定义一些PDF Meta-Data。
%    \begin{macrocode}
\renewcommand{\maketitle}{%
  \GetFileInfo{seuthesis-2019.cls}
  \hypersetup{
    pdfauthor={\seut@var@str@titlepage@author},
    pdfcreator={\fileversion}
  }
%    \end{macrocode}
% 定义封面页。首先空出44pt(来自于Word模版内空行)的距离,放置东南大学文字Logo以及总标题。
%    \begin{macrocode}
  \begin{titlepage}
    \vspace*{44pt}
    \begin{centering}
      \includegraphics[width=7.2cm]{resources/seu-text-logo}\\ \relax
      \vspace{34pt}
      {\CJKfamily{zhsong}\zihao{-0}\textbf{\seut@const@str@titlepage@logotitle}}\\ \relax
    \end{centering}
%    \end{macrocode}
% 空开46pt(Word内空行),放置论文题目。
%    \begin{macrocode}
    \vspace{46pt}
    {\centering
    \heiti\zihao{2}
    \renewcommand*{\arraystretch}{1.65}
    \begin{tabular}{p{2.5cm}c}
      \makebox[2.5cm][r]{\seut@field@str@titlepage@title}
      & \seutplaceholder[12.5cm]{\seut@var@str@titlepage@title@up} \\ \relax
      \ifcsempty{seut@var@str@titlepage@title@down}{}{
      & \seutplaceholder[12.5cm]{\seut@var@str@titlepage@title@down}}
    \end{tabular}
    \par % to make centering work
    }

%    \end{macrocode}
% 空开46pt(Word内空行),放置封面其他信息。
%    \begin{macrocode}
    \vspace{46pt}
    {\centering
    \zihao{-2}
    \renewcommand*{\arraystretch}{1.4}
    \begin{tabular}{p{3cm}c}
      \makebox[3cm][r]{\seut@field@str@titlepage@studentnum}
      & \seutplaceholder[8cm]{\seut@var@str@titlepage@studentnum} \\ \relax
      \makebox[3cm][r]{\seut@field@str@titlepage@author}
      & \seutplaceholder[8cm]{\seut@var@str@titlepage@author} \\ \relax
      \makebox[3cm][r]{\seut@field@str@titlepage@department}
      & \seutplaceholder[8cm]{\seut@var@str@titlepage@department} \\ \relax
      \makebox[3cm][r]{\seut@field@str@titlepage@major}
      & \seutplaceholder[8cm]{\seut@var@str@titlepage@major} \\ \relax
      \makebox[3cm][r]{\seut@field@str@titlepage@supervisor}
      & \seutplaceholder[8cm]{\seut@var@str@titlepage@supervisor} \\ \relax
      \makebox[3cm][r]{\seut@field@str@titlepage@period}
      & \seutplaceholder[8cm]{\seut@var@str@titlepage@period}
    \end{tabular}
    \par % to make centering work
    }
  \end{titlepage}
}
%    \end{macrocode}
% \subsection{声明页}

% \subsection{特殊章节}

%</cls>
% \Finale

% \iffalse
% TODO: style
%<*doc>
\ProvidesPackage{seuthesisdoc}
%</doc>
% \fi
\endinput